%\documentclass[12pt,cancelspace]{exam}
\doucumentclass[12pt,answers]{exam}
\usepackage{color,geometry}
\usepackage{verbatim}
\usepackage{amsmath,amssymb}
\usepackage{graphicx}
\geometry{hmargin={1in,1in},vmargin={1in,1in}}
\begin{document}

\section*{Multiple Choice: } 
\begin{questions} 
%%%%% Question 1: 
\question What is the correct command to remove a file? 
\begin{description} 
\item[A.] Possible Answer \verb' remove "filename"' 
\item[B.] Possible Answer \verb' rm "filename"' 
\item[C.] Possible Answer \verb' "filename" remove' 
\item[D.] Possible Answer \verb' abracadabra' 
%answer: B 
\end{description} 
\begin{solution} 
(Answer B): This is the correct command to rm a file, but if it were a directory it would be a little different.
\end{solution}

%%%%% Question 2:
\question What command creates an empty file?
\begin{description}
\item[A.] Possible Answer \verb' nano "filename"'
\item[B.] Possible Answer \verb' Emacs "filename"'
\item[C.] Possible Answer \verb' touch "filename"'
\item[D.] Possible Answer \verb' vim "filename"
%answer: C
\end{description}
\begin{solution}
(Answer C): When using the command 'touch "filename"', there will be a blank file created under that file name
\end{solution}

%%%%% Question 3:
\question What must you do to a bash file once you have finished editing it?
\begin{description}
\item[A.] Possible Answer \verb' make it executable'
\item[B.] Possible Answer \verb' make it readable'
\item[C.] Possible Answer \verb' keep editing it'
\item[D.] Possible Answer \verb' phone a friend'
%answer: A
\end{description}
\begin{solution}
(Answer A): After editing a bash file you must make it executable by using the chmod 777 'filename.sh command.
\end{solution}

%%%%% Question 4:
\question What command is used to put a repository on your server?
\begin{description}
\item[A.] Possible Answer \verb' git "repository link"'
\item[B.] Possible Answer \verb' get clone "repository link"'
\item[C.] Possible Answer \verb' git copy "repository link"'
\item[D.] Possible Answer \verb' git clone "repository link"'
%answer: D
\end{description}
\begin{solution}
(Answer D): The proper way to 'clone' a repository to your server is to use 'git clone' and then a copy of the repository link
\end{solution}

\end{questions}
\section*{Short Answer}
What are the steps to create a LaTex file and turn it into a pdf? 
\begin{solution}
First you must create your LaTex file that will become a pdf.

\verb' nano LaTex.tex'

Then you must add the headings for your pdf file to output how you would like and what packages you would like to use in your file.

\verb' example: \documentclass[12pt,answers]
                \usepackage{color, geometry}
                \usepackage verbatim
                \geometry{hmargin={1in,1in},vmargin={1in,1in}}
                \begin{document}

                *insert what you would like in your pdf.*

                \end{document}'

Save your LaTex file

\verb' Ctrl + x, y, then Enter'

Then you make the LaTex file a pdf.

\verb' pdflatex LaTex.tex'

Then a LaTex.pdf file will be created in addition to a couple other files. That is how a LaTex file is converted to a pdf file.
 
\end{solution}
\end{document}
